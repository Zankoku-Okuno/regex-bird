% !TEX TS-program = pdflatex
% !TEX encoding = UTF-8 Unicode
\documentclass[11pt]{amsart}

\usepackage[utf8]{inputenc}

\usepackage{geometry} % to change the page dimensions
\geometry{a4paper} % or letterpaper (US) or a5paper or....


\usepackage{booktabs} % for much better looking tables
\usepackage{array} % for better arrays (eg matrices) in maths
\usepackage{paralist} % very flexible & customisable lists (eg. enumerate/itemize, etc.)
\usepackage{verbatim} % adds environment for commenting out blocks of text & for better verbatim
\usepackage{subfig} % make it possible to include more than one captioned figure/table in a single float


\usepackage{mathtools}


\title{Parsing Regex Backreferences with Derivatives}
\author{}
%\date{} % Activate to display a given date or no date (if empty),
         % otherwise the current date is printed 

\begin{document}
\maketitle

For now, I'm just typesetting the derivative rules.

Some notations:
\begin{itemize}
\item $\Sigma$ is the set of characters in the alphabet
\item $c$ is a metavariable for a single character from $\Sigma$
\item $\Sigma^*$ is a (possibly empty) string of characters
\item $\varepsilon$ is the empty string
\item $x$ is a metavariable for the name of a capturing group/backreference
\item $\theta$ is a metavariable for a finite map from variables to strings
\item right-baiased merge of two finite maps is $A + B$
\end{itemize}

The basic idea is that we extend the accepts-empty function $\nu(r)$ from returning a simple boolean to returning possible substitutions.
Then, $\partial$ is parameterized by a substitution.
Annoyingly (but it's not that bad), $\nu$ must also be parameterized by a substitution.

Here are the core rules:

$$
\begin{array}{rcl}
\nu^\theta(\bot) &=& \{\} \\
\nu^\theta(c) &=& \{\} \\
\nu^\theta(\varepsilon) &=& \{\theta\} \\
\nu^\theta(r \cdot r') &=&
	\bigcup_{\theta' \in \nu^\theta(r)}
		\nu^{\theta'}(r') \\
\nu^\theta(r + r') &=& \nu^\theta(r) \cup \nu^\theta(r') \\
\nu^\theta(r^*) &=& \{\theta\} \\
\nu^\theta(x{=}\Sigma^*.\,r) &=&
	\{ \theta' + \{x \mapsto \Sigma^*\}
	\mid \theta' \in \nu^\theta(r)
	\} \\
\nu^\theta(x) &=&
	\begin{cases}
	\{\theta\} & \textrm{if }\theta(x) = \varepsilon \\
	\{\} & \textrm{otherwise}
	\end{cases} \\
\nu^\theta(\theta'{:r}) &=& \nu^{\theta + \theta'}(r) \\
\end{array}
$$

$$
\begin{array}{rcl}
\partial_c^\theta \bot &=& \bot \\
\partial_c^\theta c' &=&
	\begin{cases}
	\varepsilon & \textrm{if } c = c' \\
	\bot & \textrm{otherwise}
	\end{cases} \\
\partial_c^\theta \varepsilon &=& \bot \\
\partial_c^\theta (r\cdot r') &=&
	\partial_c^\theta r\cdot r' +
	\sum_{\theta' \in \nu^\theta(r)} \theta'{: \partial_c^{\theta+\theta'} r'} \\
\partial_c^\theta (r + r') &=& \partial_c^\theta r + \partial_c^\theta r' \\
\partial_c^\theta r^* &=& \partial_c^\theta r\cdot r^* \\
\partial_c^\theta x{=}\Sigma^*.\,r &=& x{=}\Sigma^*c.\,\partial_c^\theta r \\
\partial_c^\theta x &=&
	\begin{cases}
	\partial_c^\theta \Sigma^* & \textrm{where }\theta(x)=\Sigma^* \\
	\bot & \textrm{if } x \notin \mathrm{dom}(\theta) \\
	\end{cases} \\
\partial_c^\theta \theta'{: r} &=& \partial_c^{\theta + \theta'}r \\
\end{array}
$$


\end{document}
